\documentclass[12pt,a4paper]{article}
\usepackage[utf8]{inputenc}
\usepackage{amsmath}
\usepackage{amsfonts}
\usepackage{amssymb}
\usepackage{multicol}

% bibliography
\usepackage[authoryear]{natbib}

% figures
\usepackage{graphicx}
\usepackage{subcaption}
\usepackage{wrapfig}

%text format
\usepackage{color}
\usepackage{listings}

% document propoerties format
\usepackage[margin=0.5in]{geometry}
\usepackage{fancyhdr}
\usepackage{indentfirst}
\usepackage{multicol}

% newcommands for this document
\newcommand{\dudx}{\frac{\partial u}{\partial x}}
\newcommand{\dudy}{\frac{\partial u}{\partial y}}
\newcommand{\dvdx}{\frac{\partial v}{\partial x}}
\newcommand{\dvdy}{\frac{\partial v}{\partial y}}

\newcommand{\dTdx}{\frac{\partial T}{\partial x}}
\newcommand{\dTdy}{\frac{\partial T}{\partial y}}

\newcommand{\dTpdx}{\frac{\partial T'}{\partial x}}
\newcommand{\dTpdy}{\frac{\partial T'}{\partial y}}

\begin{document}
\subsection*{Variables:}
\begin{table}[h!]
\begin{tabular}{ll}
 $C_D^*$ &  Reference drag coefficient (1E-3) \\
 $\rho_a$ &  air density [kg m$^{-3}$] \\
 $T_C$ & temperature field in ($y$) without eddies in [K]\footnote{The CTRL field in Foussard et al. 2019}  \\
 $T'$ & temperature perturbation (eddies only) in ($x,y$) in [K]\footnote{The EDDY field in Foussard et al. 2019}  \\
  $T$ & total temperature field ($=T_C + T'$) in ($x,y$) in [K]  \\
 $\vec{u}$ & velocity vector with components $(u,v)$ in [m/s] \\
 $x,y$ & spatial coordinates [m]
\end{tabular}
\end{table}

\subsection*{stress calculations}
For tuning parameter $\alpha$, the surface stress is 
\begin{align*}
\tau = \rho_a C_D^*(1+\alpha T')\vec{u}\lVert u \rVert
\end{align*}

The divergence of the stress is 

\begin{align*}
\begin{split}
\nabla \cdot \tau = C_D^* \rho_a \left( \frac{(1+\alpha T')(4\dudx u^3 + 2 \dvdx u^2 v + 2 \dudx u v^2)}{2\sqrt{u^4+u^2v^2}}+ \frac{(1+\alpha T')(4\dvdy v^3 + 2 \dudy u v^2 + 2 \dvdy u^2)}{2\sqrt{v^4+u^2v^2}}\right.+...\\
\left. \alpha\dTpdx\sqrt{u^4+u^2v^2}-\alpha\dTpdy\sqrt{v^4+u^2v^2} \right)
\end{split}
\end{align*}

The curl of the stress is 

\begin{align*}
\begin{split}
\nabla \times \tau = C_D^* \rho_a \left( \frac{(1+\alpha T')(4\dudy u^3 + 2 \dvdy u^2 v + 2 \dudy u v^2)}{2\sqrt{u^4+u^2v^2}}- \frac{(1+\alpha T')(4\dvdx v^3 + 2 \dudx u v^2 + 2 \dvdx u^2)}{2\sqrt{v^4+u^2v^2}}\right.+...\\
\left. \alpha\dTpdy\sqrt{u^4+u^2v^2}-\alpha\dTpdx\sqrt{v^4+u^2v^2} \right)
\end{split}
\end{align*}

The gradient of the sea surface temperature in the "along-wind" or the "downwind" direction is

\begin{align*}
\nabla SST_{\parallel} = \left( \frac{u\left( \frac{\dTdx u}{\sqrt{u^2+v^2}} + \frac{\dTdy v}{\sqrt{u^2+v^2}}\right)}{\sqrt{u^2+v^2}} ,  \frac{v\left( \frac{\dTdx u}{\sqrt{u^2+v^2}} + \frac{\dTdy v}{\sqrt{u^2+v^2}}\right)}{\sqrt{u^2+v^2}} \right)
\end{align*}

The gradient of the sea surface temperature in the "cross-wind" direction is

\begin{align*}
\nabla SST_{\bot} = \left(\dTdx- \frac{u\left( \frac{\dTdx u}{\sqrt{u^2+v^2}} + \frac{\dTdy v}{\sqrt{u^2+v^2}}\right)}{\sqrt{u^2+v^2}} , \dTdy- \frac{v\left( \frac{\dTdx u}{\sqrt{u^2+v^2}} + \frac{\dTdy v}{\sqrt{u^2+v^2}}\right)}{\sqrt{u^2+v^2}} \right)
\end{align*}

\newpage

\subsection*{Taking Limits}

\subsubsection*{In the limit that $\vec{u} = (\bar{u},0)$ everywhere}

This means that $\dudx = \dudy = \dvdx = \dvdy = v = 0$ and $u=\bar{u}$

So the above quantities become:\\
Divergence:
\begin{align*}
\nabla \cdot \tau = C_D^* \rho_a \alpha\bar{u}^2\dTpdx
\end{align*}
Down-wind gradient:
\begin{align*}
\nabla SST_{\parallel} &= \left( \dTdx , 0 \right) \\
&= \left( \frac{\partial}{\partial x}(T_C + T') , 0) \right)\\
&= \left( \dTpdx, 0 \right)
\end{align*}

Which, when plotted against eachother, will have a slope of $ C_D^* \rho_a \alpha\bar{u}^2$.\\
\vspace*{0.5in}

Curl:
\begin{align*}
\nabla \times \tau = -C_D^* \rho_a \alpha\bar{u}^2\dTpdy
\end{align*}
Cross-wind gradient:
\begin{align*}
\nabla SST_{\bot} &= \left(\dTdx- \dTpdx , \dTdy- 0 \right)\\
\nabla SST_{\bot} &= \left(0 , \frac{\partial}{\partial y}(T_C + T') \right)\\
\nabla SST_{\bot} &= \left(0 , \frac{\partial}{\partial y}T_C+ \frac{\partial}{\partial y}T' \right)
\end{align*}

Which, when plotted against eachother, will have a slope of approximately $ -C_D^* \rho_a \alpha\bar{u}^2$ if $\frac{\partial T_C}{\partial y}$ is small relative to $\dTpdy$.

\subsubsection*{In the limit that the velocity field is determined by a coupling coefficient }
 $\vec{u} = (\bar{u} + T' \gamma, 0)$ for a coupling coefficient $\gamma$ and the drag coefficient is nolonger a function of temperature (i.e. $\alpha = 0$)

The divergence of the stress is now

\begin{align*}
\nabla \cdot \tau =& C_D^* \rho_a \left( \frac{(4\dudx u^3)}{2u^2} \right)\\
\nabla \cdot \tau =& C_D^* \rho_a \left( 2u\dudx \right)\\
\nabla \cdot \tau =& C_D^* \rho_a \left( 2u\gamma\dTpdx  \right)\\
\nabla \cdot \tau =& C_D^* \rho_a \left( 2\gamma\dTpdx (\bar{u} + \gamma T') \right)
\end{align*}

and the gradient of the sea surface temperature in the "downwind" direction is 

\begin{align*}
\nabla SST_{\parallel} = \left( \dTpdx, 0 \right)
\end{align*}
If $\gamma T'$ is small with respect to $\bar{u}$, then the slope of the divergence plotted against the "down-wind" SST gradient will be $C_D^*\rho_a 2\gamma\bar{u}$. \\

The curl of the stress is now

\begin{align*}
\nabla \times \tau =& C_D^* \rho_a \left( \frac{(4\dudy u^3 )}{2u^2} \right)\\
\nabla \times \tau =& C_D^* \rho_a \left(  2u\dudy \right)\\
\nabla \times \tau =& C_D^* \rho_a \left(  2u\gamma\dTpdy \right)\\
\nabla \times \tau =& C_D^* \rho_a \left( 2\gamma\dTpdy (\bar{u} + \gamma T') \right)
\end{align*}

and the gradient of the sea surface temperature in the "cross-wind" direction is 

\begin{align*}
\nabla SST_{\bot} &= \left(0 , \frac{\partial}{\partial y}T_C+ \frac{\partial}{\partial y}T' \right)
\end{align*}

so in the limit that  $\gamma T'$ is small with respect to $\bar{u}$ and $\frac{\partial}{\partial y}T_C$ is small with respect to $\frac{\partial}{\partial y}T'$, the resulting slope of these two quantities plotted against eachother would also be $C_D^* \rho_a 2\gamma\bar{u}$. \\

\end{document}