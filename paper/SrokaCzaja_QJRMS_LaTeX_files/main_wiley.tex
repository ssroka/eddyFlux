%%%%%%%%%%%%%%%%%%%%%%%%%%%%%%%%%%%%%%%%%%%%%%%%%%%%%%%
% A template for Wiley article submissions.
% Developed by Overleaf. 
%
% Please note that whilst this template provides a 
% preview of the typeset manuscript for submission, it 
% will not necessarily be the final publication layout.
%
% Usage notes:
% The "blind" option will make anonymous all author, affiliation, correspondence and funding information.
% Use "num-refs" option for numerical citation and references style.
% Use "alpha-refs" option for author-year citation and references style.

\documentclass[alpha-refs]{wiley-article}
% \documentclass[blind,alpha-refs]{wiley-article}


%%%%%%%%%%%%%%%%%%%%%%%%%%%%%%%%%%%%%%%%%%%%%%%%%%%%%%%%%
%%%%%%%%%%% JDT:  Annotation Code %%%%%%%%%%%%%%%%%%%%%%%%%%%%%%%%%%
%%%%%%%%%%%%%%%%%%%%%%%%%%%%%%%%%%%%%%%%%%%%%%%%%%%%%%%%%

\usepackage{color}
\usepackage{ulem}

 % Uncomment to display with annotation; comment out otherwise
\newcommand{\add}[1]{\textcolor{blue}{#1}}
\newcommand{\delete}[1]{\textcolor{red}{\sout{#1}}}
\newcommand{\edit}[2]{\textcolor{red}{\sout{#1}} \textcolor{blue}{#2}}
\newcommand{\mnote}[1]{\marginpar{\textcolor{green}{\textbf{#1}}}}

\newcommand{\beginsupplement}{%
\setcounter{table}{0}
        \renewcommand{\thetable}{S\arabic{table}}%
        \setcounter{figure}{0}
        \renewcommand{\thefigure}{S\arabic{figure}}%
     }
 % Uncomment to display without annotation; comment out otherwise
%\newcommand{\add}[1]{#1}
%\newcommand{\delete}[1]{}
%\newcommand{\edit}[2]{#2}
%\newcommand{\mnote}[1]{}

%%%%%%%%%%%%%%%%%%%%%%%%%%%%%%%%%%%%%%%%%%%%%%%%%%%%%%%%%
%%%%%%%%%%%%%%%%%%%%%%%%%%%%%%%%%%%%%%%%%%%%%%%%%%%%%%%%%
%%%%%%%%%%%%%%%%%%%%%%%%%%%%%%%%%%%%%%%%%%%%%%%%%%%%%%%%%

% Add additional packages here if required
\usepackage{siunitx}

% Update article type if known
%\papertype{Original Article}
% Include section in journal if known, otherwise delete
%\paperfield{Journal Section}

\title{Wintertime cooling of the Kuroshio: experimenting with simple air-sea interaction models}

% List abbreviations here, if any. Please note that it is preferred that abbreviations be defined at the first instance they appear in the text, rather than creating an abbreviations list.
%\abbrevs{ABC, a black cat; DEF, doesn't ever fret; GHI, goes home immediately.}

% Include full author names and degrees, when required by the journal.
% Use the \authfn to add symbols for additional footnotes and present addresses, if any. Usually start with 1 for notes about author contributions; then continuing with 2 etc if any author has a different present address.
\author[1]{Sydney Sroka}
\author[2]{Arnaud Czaja}

%\contrib[\authfn{1}]{Equally contributing authors.}

% Include full affiliation details for all authors
\affil[1]{Department of Mechanical Engineering, Massachusetts Institute of Technology, Cambridge, Massachusetts, 02139, USA}
\affil[2]{Imperial College, Department of Physics, Prince Consort Road, London SW7 2AZ, United Kingdom}

\corraddress{Sydney Sroka, Department of Mechanical Engineering, Massachusetts Institute of Technology, Cambridge, Massachusetts, 02139, USA}
\corremail{ssroka@mit.edu}

%\presentadd[\authfn{2}]{Department, Institution, City, State or Province, Postal Code, Country}

\fundinginfo{MIT Office of Graduate Education Fellowship}

% Include the name of the author that should appear in the running header
\runningauthor{Sroka and Czaja}

%--------preamable--------

\usepackage[utf8]{inputenc}
\usepackage{amsmath}
\usepackage{amsfonts}
\usepackage{amssymb}
\usepackage{multicol}

% figures
\usepackage{graphicx}
\usepackage{subcaption}
\usepackage{wrapfig}
\usepackage[section]{placeins} % keeps floats inside this sections

% document properties format

\usepackage{indentfirst}
% tables
\usepackage{booktabs}
\setlength{\heavyrulewidth}{1.5pt}
\setlength{\abovetopsep}{4pt}
\usepackage{adjustbox}
\usepackage{multirow}
% equations
\usepackage{cancel} % use \cancelto{0}{x} to draw an arrow thru x with a zero
\usepackage{tikz}

% new commands
\newcommand{\Vmag}[1]{\| \mathbf{#1}\|}
% newcommands for this document
\newcommand{\Ub}{\overline{U}\hspace*{1mm}}
\newcommand{\Up}{U'}
\newcommand{\Dhb}{\overline{\Delta h}\hspace*{1mm}}
\newcommand{\Dhp}{\Delta h'}
\newcommand{\To}{T_o\hspace*{1mm}}
\newcommand{\Top}{T_o'}
\newcommand{\Ta}{T_a\hspace*{1mm}}
\newcommand{\Tap}{T_a'}
\newcommand{\qo}{q_o\hspace*{1mm}}
\newcommand{\qop}{q_o'}
\newcommand{\qa}{q_a\hspace*{1mm}}
\newcommand{\qap}{q_a'}
\newcommand{\CDs}{C_D^s}
\newcommand{\CDL}{C_D^L}
\newcommand{\cpa}{c_p^{air}}
\usepackage{tikz}
\usepackage{url}
\usepackage{hyperref}



\begin{document}

\maketitle

\begin{abstract}
    Western boundary currents exhibit significant meandering on sub-seasonal timescales and wavelenghts of a few hundreds of kilometers, but the extent to which the ocean\\ mesoscale eddy activity associated with these motions can enhance the turbulent air-sea heat flux averaged over a large spatial domain has not been fully characterized. This study aims to quantify the enhancement of the winter time mean sensible and latent heat flux from ocean mesoscale eddies in reanalysis data. Results from a correlation analysis between eddy activity, using sea surface height anomaly data as a proxy, and the winter time mean heat fluxes do not suggest that there are broad correlations from winter-to-winter over the 2003-2018 period. The results from simple models designed to isolate the contributions from the eddies suggest that the eddy-enhancement of the heat flux is small compared to the long-time, large-spatial scale mean in the reanalysis data, perhaps due to a lack of mesoscale sea surface temperature activity in the reanalysis data.

% Please include a maximum of seven keywords
\keywords{air-sea interaction, ERA5, rectified effect, ocean mesoscale eddies}
\end{abstract}


\section{Introduction\label{Sec:intro}}
% it looks like QJRMS might be free if the page count is under 17.

The western boundary currents of the World Ocean provide an important heat source for the atmospheric storm-track \citep{hoskins1990existence}. Regions with strong currents like the Kuroshio Extension (KE) or Gulf Stream can be expected to modify the wind stress by as much as 20\%, which has significant implications for the air-sea heat flux in those regions \citep{Chelton2004}. This study focuses on the North Pacific and specifically the Kuroshio in winter, when the atmospheric eddy driven jet is closest to the path of the Kuroshio and its extension into the interior of the Pacific \citep{nakamura2004observed}. 




While large-scale (greater than 1000\,km) correlations between sea surface temperature (SST) and surface winds are generally negative, observations and simulations have confirmed a strong positive correlation at the mesoscale \citep{chelton2010coupled}. \citet{Chelton2004}, and subsequent studies including \citet{ONeill2010}, identified persistent, small-scale structures from high-resolution (25\,km) satellite measurements of surface winds and found that there were many features with characteristic wavelengths less than 30$^{\circ}$ in longitude and 10$^{\circ}$ in latitude which could be attributed to SST variations. The authors showed how the curl and divergence of the 4-year averaged wind field were directly proportional to the downwind and crosswind gradient of the SST field, respectively.\par

\citet{putrasahan2013isolating} isolated the influence of ocean mesoscale eddies on surface winds in the KE region by conducting companion simulations with a regional model where eddies are either resolved or filtered out. Their results showed that the eddies play an important role in driving the surface wind through two mechanisms, vertical mixing from alternating regions of boundary layer stability and sea level pressure anomalies. The strength of the former mechanism, which is measured by the magnitude of the linear scaling between the wind stress divergence (curl) and downwind (crosswind) SST gradient, was found to exhibit a strong seasonality being much larger in the winter, when the atmosphere is generally less stable, than the summer. The simulations exhibited a steeper linear relationship than was calculated from satellite observations, but the findings strongly supported the conclusion that the mesoscale eddies are an important driver of surface wind stress. With regard to the latter pressure-driven mechanism, which is measured by the linear scaling of the wind convergence and the sea level pressure Laplacian, the authors found that the presence of eddies significantly strengthened the ocean-atmosphere coupling compared to the eddy-free simulation.\par


It has recently been suggested that the meanders or eddies that develop on western boundary currents with characteristic length scales of a few 100\,km have a so called \textit{rectified effect} on the turbulent air-sea heat flux ($Q$) because a little bit more heat is lost over a warm anticyclone than is gained over a cold cyclone \citep{SMALL2008274,foussard2019storm}. This then was suggested to have a significant effect on the storm track, at least in the North Pacific \citep{ma2015improved,ma2017revised} and South Atlantic \citep{villas2015signature}. \citet{SMALL2008274} reviewed various physical mechanisms that facilitate air-sea heat flux from ocean eddies and fronts including: the destabilizing effect of air traveling over SST gradients, large eddies increasing the boundary layer depth, secondary circulations associated with spatial changes in the pressure-gradient force, and the way changes in ocean surface velocities can locally increase the surface stress. The source of SST anomalies near strong SST frontal regions like the KOE region is thought to be oceanic weather rather than atmospheric forcing \citep{bishop2017scale}.\par 

 

Isolating the effect of ocean mesoscale eddies on the turbulent heat fluxes from observations is difficult because, unlike in an atmospheric general circulation model (AGCM), one cannot separately analyse a time history of the heat flux with and without eddies, as was done in \citet{putrasahan2013isolating} or \citet{foussard2019storm} for example. In this paper we propose a new methodology which attempts to do so by developing simplified models of air-sea interactions tuned to realistic air-sea heat fluxes. The models use just one or two eddy coupling coefficients, so it is straightforward to evaluate the effect of the eddies on the time mean and spatially averaged $Q$. The first model represents the eddy-enhancement as an increase in the surface wind speed, and is based on the work from \citet{ONeill2010} who used wind vector and SST observations to estimate the wind vector magnitude and direction response to the persistent mesoscale eddies near western boundary currents. The authors found that air flow along a positive (negative) SST gradient led to an acceleration (deceleration) and anticyclonic (cyclonic) changes in the wind velocity; the magnitudes of changes in the wind velocity and direction were about 1-2\,ms$^{-1}$ and 4$^{\circ}$-8$^{\circ}$. The second model adds a mechanism to the first model such that the eddy-enhancement can also be represented through an increase in the sea surface drag following \citet{behringer1979thermal}. \citet{SMALL2008274} discuss several studies that consider the mechanism by which warmer SSTs lead to atmospheric instabilities which increase the local sea surface drag coefficient and therefore the stress. The emerging consensus appears to be that an increase in the sea surface drag with SST is a much smaller contribution to the increase in surface stress compared to other mechanisms like a pressure-gradient driven increase in the surface wind speed. 


  





The paper is structured as follows. We introduce the data used for the air-sea interaction analysis in section \ref{Sec:DataMethods}, and discuss the impact of the mesoscale ocean eddy activity on the time and spatial mean $Q$ in section \ref{Sec:Corr}. Section \ref{Sec:Param} introduces the interaction models and shows that the models are able to well-represent the time mean air-sea turbulent heat flux over the KE in winter. Finally, a summary is offered in section \ref{Sec:Conclusion}.  



\section{Data and Methods\label{Sec:DataMethods}}
The ERA5 reanalysis data (\citet{ERA5_data,hersbach2020era5}; downloaded from the Copernicus Climate Change Service (C3S) Climate Data Store) was extracted at a 0.25$^{\circ}$ spatial resolution and at a 6-hour temporal resolution for the boreal winter months December through March (DJFM). The region considered is a patch of the Northwest Pacific Ocean (30$^\circ$N to 41.5$^\circ$N,142.5$^\circ$E to 169$^\circ$E) that captures the major features of the Kuroshio Extension. The reanalysis fields include the sea surface temperature ($\To$), the temperature at 2\,m ($\Ta$), the surface pressure ($p_0$), the dew point temperature at 2\,m ($T_d$), the surface wind vector ($[u,v]$), the surface sensible heat flux ($Q_s$) and the surface latent heat flux ($Q_L$). \par 
A 2D FFT filter was used to separate the high spatial frequency content from the low spatial frequency content in the field variables from the reanalysis data for each of the models. The filtering scheme used here is similar to the one used in \citet{scott2005direct}. First, a bilinear plane is calculated with a least squares fit to $c_0 + c_1 x+c_2y$, where $c_i$ variables are constants and $x$ and $y$ are the zonal and meridional directions, respectively. This plane is subtracted from the field to spatially de-mean and de-trend the data. Next, a 2D FFT is applied. The cutoff frequency is the radius of a circle centered on the origin of the transformed field, and the low-pass (high-pass) field is recovered by removing all of the frequency content outside (inside) of this circle and then inverting the transform. A filter length scale of 500\,km, or spatial frequency of 0.002\,km$^{-1}$, was used for all field variables to separate the eddy length scales from the large-scales. The filter is applied to each time point before any time-averaging is done. The bilinear plane is added to the low-pass field after the inverse transform step. High-pass fields are denoted with primes $\bullet’$, and low-pass fields are denoted with overbars $\overline{\bullet}$. \par 
This filtering procedure leads to several important properties of the filtered output. It ensures that the original signal is equal to the sum of its low-pass and high-pass components ($\bullet = \overline{\bullet} + \bullet’$), that low-pass filtering a single high-pass filtered field vanishes ($\overline{\bullet'}  = 0$), and that low-pass filtering a field that was already low-pass filtered does not change the output $\left(\overline{\left(\overline{\bullet}\right)}  = \overline{\bullet}\right)$. However, it is important to note that low-pass filtering the product of a low-pass filtered field and a high-pass filtered field does not guarantee the result vanishes ($\overline{\overline{\bullet}\bullet'}\neq 0$); this can be illustrated with a simple, 1D example. If a signal $s(x)$, where $x$ measures horizontal distance, is the sum of two cosine functions with wavenumbers $k_1$ and $k_2$, and the cutoff wavenumber $k_c$ is such that $k_1 > k_c > k_2$, then
\begin{equation}
s(x) = \cos(k_1x)+\cos(k_2x)
\end{equation}
After applying the spectral filter described above, we obtain:
\begin{equation}
\overline{s}(x) = \cos(k_2x) \; \mbox{ and } s'(x) = \cos(k_1x),    
\end{equation}
and as a result:
\begin{equation}
\overline{s}(x)s'(x) = \cos(k_2x) \cos(k_1x) = \frac{ \cos((k_1+k_2)x) + \cos((k_1-k_2)x) }{2}
\end{equation}



Since $k_1+k_2$ is guaranteed to be greater than $k_c$, whether $\overline{\overline{s} s'}$ vanishes depends on whether $\lvert k_1-k_2 \rvert$ is greater than or less than $k_c$. In an atmospheric context, extinguishing $\overline{\overline{s} s'}$ is guaranteed by considering the zonal mean (i.e. $k_c=0$). For the regional reanalysis datasets considered here however, no products of high-passed and low-passed fields are expected to vanish after low-pass filtering due to the broad range of wavenumber content around the eddy scale \citep{scott2005direct,tulloch2011scales}.

\section{Correlations Between the Sea Surface Height Anomaly Contour Length and the Turbulent Heat Flux\label{Sec:Corr}}

This analysis uses the length of the sea surface height anomaly contours as a proxy for the amount of eddy activity in a given winter. As shown in \citet{qiu2005variability} and \citet{qiu2020reset}, the sea surface height anomaly (SSHA) contours indicate the Kuroshio's path. \citet{qiu2020reset} labels the Kuroshio's path for all years from 1992 through 2019 as either stable, when there was very little change in the contours' locations, or unstable, when there were large fluctuations in the SSHA contours throughout the year. In unstable years, the contours' large fluctuations result in a longer average length and much more curvature compared to those from stable years, as seen in figure 2 of \citet{qiu2020reset}. Years where the trajectories of the contours are more dynamic are also associated with increased eddy activity, as the scale and locations of the mesoscale SST anomalies are reflected in the curvature of the SSHA contours. The SSHA data used in this analysis is from the E.U. Copernicus Marine Service Information \citep{SSHA_data,SSHA_data_doc}. \par 


The results of a correlation analysis do not suggest that increased eddy activity through the rectified effect discussed in section \ref{Sec:intro} leads to a greater winter time mean heat flux in the reanalysis data. A representative contour plot of SSHA is shown in figure \ref{fig:ssha_HF}a. The 0.4\,m contour, highlighted in white in figure \ref{fig:ssha_HF}a, was chosen for its proximity to the largest SST gradient, which indicates the main path of the Kuroshio. Figure \ref{fig:ssha_HF}b shows a scatterplot of the winter time mean turbulent heat flux as a function of the length of the 0.4\,m contour. The length of the contour level was calculated at daily intervals and then averaged over each DJFM winter. The time mean heat flux for each winter in the region is computed from the reanalysis sensible and latent surface heat flux fields.  As expected, the length of the 0.4\,m contour in unstable years (blue crosses) are nearly all longer than those of stable years (red crosses). The Pearson correlation coefficients between the contour length and the winter time mean heat fluxes for the stable years, unstable years, and all years are -0.33, -0.11, and -0.18, respectively. The correlation coefficients are all slightly negative, which is reflected in the linear fits shown in figure \ref{fig:ssha_HF}b, but none are found to be statistically significant. Though not statistically significant, a negative correlation would suggest that an increase in eddy activity decreases the heat flux rather than enhances it, as the rectified effect hypothesis would predict \citep{ma2016western,ma2017revised}. To investigate this result, two simple models of turbulent heat flux are constructed to isolate the eddy contribution in the reanalysis data. 

% p values for 
% red: 0.35
% blue: 0.83
% black: 0.49



\begin{figure}[tb]
\begin{tikzpicture}
    \draw (0, 0) node[inner sep=0] {\includegraphics[width=0.5\textwidth]{imgs/ssh_cntr_2003.pdf}};
    \draw (2.3,1.5) node[fill=white] {\scriptsize a)};
\end{tikzpicture}
\begin{tikzpicture}
    \draw (0, 0) node[inner sep=0] {\includegraphics[width=0.5\textwidth]{imgs/ssh_length_vs_HF_box3_beta_2003_2018_5_redblue.pdf}};
    \draw (2.7,1.5) node {\scriptsize b)};
\end{tikzpicture}
\caption{a) Sea surface height anomaly in units of meters. b) Scatterplot of the average winter time mean turbulent heat flux of the region shown in a) as a function of the length of the 0.4\,m contour in the same region. The slopes of the linear fits (m) for the stable (red), unstable (blue), and all (black) years are indicated in the legend. \label{fig:ssha_HF}}
   
\end{figure}














\section{Model Results and Discussion\label{Sec:Param}}
\subsection{The $\beta$-model}
The first model, the $\beta$-model, solely accounts for the observations by \citet{Chelton2004} and \citet{ONeill2010} that perturbations of surface wind speed and sea surface temperature are positively correlated on the oceanic mesoscale. If $\beta$ is the coupling coefficient measuring the strength of this relation, we write $U' = \beta T'_o$ where $U'$ is the perturbation wind speed, $T'_o$ is the high-pass filtered SST from the reanalysis data, and $\beta >0$ is a constant. Accordingly, our first model is:
\begin{align}
    Q^\beta  = \rho_a C_D (\overline{U} + \beta T_o')\Delta h\label{eq:beta_model}
\end{align}
where $Q^\beta$ is the sum of the surface sensible heat flux and surface latent heat flux, $C_D$ is a constant exchange coefficient, $\rho_a$ is the density of air, and $\overline{U}$ is low-pass filtered magnitude of the horizontal surface wind vector $\left\lVert (u,v) \right\rVert$. The moist enthalpy potential  $\Delta h$ is $c_p (T_o-T_a)+L_v(q_o^*-q_a)$, where $c_p$ is the specific heat of air, $T_o$ is the sea surface temperature, $L_v$ is the latent heat of vaporization, $q_o^*$ is the saturation specific humidity at $T_o$, and $q_a$ is the specific humidity at $T_a$ and $T_d$.\par 

% c_p is assumed to be 1000 J/kg/K
% rho_a is assumed to be 1.2

An interior-point optimization scheme is used to solve for the exchange coefficient $C_D$ and the eddy coupling coefficient $\beta$. The optimization is performed separately for each DJFM winter, yielding wintertime timeseries for the parameters $C_D$ and $\beta$. Specifically, we minimise the magnitude of the objective function $J$:
\begin{align}
J([\beta, C_D]) = 
Q^{ERA5}(x,y,t) - Q^\beta(x,y,t;[\beta,C_D]),
\end{align}
where $Q^{ERA5}(x,y,t)$ is the sum of the surface sensible and latent heat flux fields from the reanalysis data. Analogous objective functions are used for subsequent models. The mean and standard deviation of $\beta$ across all winters is found to be 0.25$\pm$0.03\,ms$^{-1}$K$^{-1}$. The values of the exchange coefficient $C_D$ are very consistent between winters, with a mean and standard deviation of (1.4$\pm$0.01)$\times10^{-3}$. \par 

The simple $\beta$-model is successful at representing the winter time mean (denoted with angle brackets $\langle\bullet\rangle$) turbulent heat flux as shown in figure \ref{fig:cmp_ERA5_beta_2003_2007}, especially in the center of the selected domain where the heat flux is largest. The results are only shown here for two distinct winters, 2003 in the top row and 2007 in the bottom row, but are consistent across all years. These two years are excellent examples of when the path of the Kuroshio did not change much (2003), and when the path exhibited many large fluctuations (2007), throughout the year. Consistent with figure 1, the larger turbulent cooling of the ocean in 2003 is associated with a meridionally narrower structure and less meandering of the Kuroshio than in 2007 (note the different colorscale in figure 2 between the upper and lower panels). Note that the errors are less than 10 \% over most of the domain, except in the northeastern and southwestern corners where they reach about 20 \%. \par 

\begin{figure}[tb]
    \centering
    \includegraphics[width=\textwidth]{imgs/cmp_model_ERA5_250_fft_box3_beta_2007.pdf}
    \caption{The time mean turbulent heat flux from ERA5 (left), the time-averaged turbulent heat flux from the $\beta$-model (center), and the point-wise relative error between the first two fields (right) for the winters of 2003 (top) and 2007 (bottom). Heat fluxes are expressed in Wm$^{-2}$. \label{fig:cmp_ERA5_beta_2003_2007} }
\end{figure}


To study the sensitivity of the model's results to changes in the $\beta$ parameter, the root-mean-squared-error (RMSE) is calculated between the time mean heat flux from ERA5 and the time mean heat flux from the model using a scaled value for $\beta$. For each year the RMSE is

\begin{align}
    RMSE = \sqrt{\frac{1}{N}\sum_{i=1}^N\langle\left(Q^{ERA5}\rangle^i-\langle Q^\beta(\gamma\beta) \rangle^i\right)^2 }
\end{align}

where $N$ is the total number of spatial points in the region of interest (the model is evaluated on the ERA5 grid, which recall has a resolution of $0.25^{\circ}$) and the scaling parameter $\gamma$ varies from -10 to 10. The RMSEs between the reanalysis and the model as a function of the scaled $\beta$ are shown in figure \ref{fig:RMSE_beta} with open circles, and each curve corresponds to a separate year. The RMSE of the unscaled model ($\gamma$=1) is shown with a filled black circle for each year. For these sensitivity tests, $C_D$ is held at its optimised value for each winter, although as previously mentioned the variance in $C_D$ among winters is quite small.\par 
These results show a clear optimality near a value of $\beta = +0.25$. This is very near the value quoted by \citet{ONeill2010} of $+0.3$ (see their figure 3a), which is particularly encouraging. The years where the Kuroshio is considered stable and unstable are again shown in red and blue, respectively. There is no clear relationship between the RMSE and the stability, reinforcing the conclusion from section \ref{Sec:Corr} that there is little connection between the wintertime turbulent air-sea heat flux and the level of mesoscale eddy activity over the Kuroshio in the ERA5 dataset. \par 


To address this issue more quantitatively, we now use the $\beta$-model to compute the contribution of the mesoscale SST anomalies to the spatially averaged (low-pass filtered) surface turbulent heat flux. Following the notation in section \ref{Sec:DataMethods}, we write:
\begin{align}
\nonumber\begin{split}
     \overline{Q^\beta}  =& \overline{\rho_a C_D (\overline{U} + \beta T_o')(\overline{\Delta h} + \Delta h')}\\
     =&\overline{\rho_a C_D \left( \overline{U}\hspace*{1mm} \overline{\Delta h} +  \overline{U}\Delta h'+\beta T_o'\overline{\Delta h}+\beta T_o'\Delta h'\right)}
\end{split}\\
  =&\underbrace{\overline{\rho_a C_D \overline{U}\hspace*{1mm} \overline{\Delta h}}}_{\overline{Q^\beta}_1} +\underbrace{\overline{\rho_a C_D \overline{U}\Delta h'}}_{\overline{Q^\beta}_2}+\underbrace{\overline{\rho_a C_D\beta T_o'\overline{\Delta h}}}_{\overline{Q^\beta}_3}+\underbrace{\overline{\rho_a C_D\beta T_o'\Delta h'}}_{\overline{Q^\beta}_4}\label{eq:expand_beta_ABC}.
\end{align}



\begin{figure}[tb]
    \centering
    \includegraphics[width=0.5\textwidth]{imgs/rms_error_250_fft_box3_beta_2018_abCDFAC__10_1.pdf}
    \caption{The root-mean-squared-error (RMSE) between the time mean turbulent heat flux from ERA5 reanalysis data and the $\beta$-model with scaled values of $\beta$. The RMSEs from years when the Kuroshio was stable are connected with red lines and those from unstable years are connected with blue lines. The RMSE using the optimal value of $\beta$ (i.e. $\gamma=1$) is shown with a filled black circle for each year.}
    \label{fig:RMSE_beta}
\end{figure}

\begin{figure}[tb]
    \centering
    \includegraphics[width=\textwidth]{imgs/ABC_L_250_fft_box3_beta_2007_abCDFAC_.pdf}
    \caption{The winter time mean of the $\beta$-model terms for 2007: the long-time, large spatial scale term $\langle\overline{Q^\beta_1}\rangle$ (a), the persistent temperature anomaly term $\langle\overline{Q^\beta_2}\rangle$ (b), and the two rectified effect terms $\langle\overline{Q^\beta_3}\rangle$ (c) and $\langle\overline{Q^\beta_4}\rangle$ (d). Note the different scales of the colorbar in the different panels.}
    \label{fig:ABC_2007}
\end{figure}

Contour plots of the winter time mean for each of the terms in equation \ref{eq:expand_beta_ABC} from 2007, which was an exceptionally unstable year, are shown in figure \ref{fig:ABC_2007}. These results show that the long-time, large-spatial scale term $\langle\overline{Q^\beta_1}\rangle$ is the dominant contributor. The eddy-enhancement terms $\langle\overline{Q^\beta_3}\rangle$ and $\langle\overline{Q^\beta_4}\rangle$ are largest on the west side of the domain where the eddy activity is most pronounced, but the magnitude of the flux from these terms is much smaller than that from $\langle\overline{Q^\beta_1}\rangle$. Recall that $\langle\overline{Q^\beta_2}\rangle$ and $\langle\overline{Q^\beta_3}\rangle$ do not vanish because the spatial frequency content of the product of a low-pass field and a high-pass field includes frequencies below the cutoff. The term $\langle\overline{Q^\beta_4}\rangle$ is the most intuitively connected to the rectified effect as it is proportional to $T'_o \Delta h'$. Figure \ref{fig:ABC_2007} indicates nevertheless that the other term proportional to $\beta$, namely $\langle\overline{Q^\beta_3}\rangle$, is adding to $\langle\overline{Q^\beta_4}\rangle$ with a similar magnitude. As emphasized in section \ref{Sec:DataMethods}, this term arises out of the filtering procedure. Together, their sum represents the total rectified effect of mesoscale eddies on the turbulent heat flux. It is shown in figure \ref{fig:ABC_comp_To_2007}a and is seen to be only about 1\,\% of $\langle\overline{Q^\beta_1}\rangle$.\par

Because the optimised value of $\beta$ for ERA5 is very close to that estimated from satellite observations, we next consider how the small magnitude of $T_o'$ in the ERA5 reanalysis data may be the reason for the small contribution from the eddy terms. The contour plot in figure \ref{fig:ABC_comp_To_2007}b of $\langle T_o'\rangle$ shows the typical distribution and magnitude of the time mean high-pass filtered SST from ERA5. Figure \ref{fig:histogram_Toprime} shows a histogram of the normalized frequency of the time mean high-pass filtered SST from each winter ($\langle T_o' \rangle$) overlaid with a normalized histogram of the 6-hourly high-pass filtered SST from all winters. The maximum and minimum values in the $\langle T_o' \rangle$ distribution are -3.6\,K and 3.0\,K, and in the synoptic distribution are -7.9\,K and 6.2\,K. While there are a few outliers, and the synoptic set has a slightly broader distribution, more than 98\% of the points in the $\langle T_o' \rangle$ distribution and more than 97\% of the points in the synoptic distribution are within $\pm$2\,K. For completeness, we note that the analysis carried out with the time mean $T'_o$ for each winter yields very similar results to that using the full SST field. It is thus the wintertime mean $T'_o$ which contributes most in ERA5 to the rectified effect.\par 
% blue, winterly 98.88% are within p/m 2K
% orange, 6-hourly 97.35% are within p/m 2K

The dynamic term of the $\beta$-model in equation \ref{eq:beta_model} is $\rho_a C_D \beta T_o'\Delta h$ and the thermodynamic term is $\rho_a C_D\overline{U}\Delta h$. The ratio of these terms, which is a measure of the turbulent heat flux that is attributable to the mesoscale eddies, is then $\beta T_o'/\overline{U}$. Since $\overline{U}$ is close to 10\,m/s throughout the domain across all years, and the average $\beta$ is 0.25, the ratio of the dynamic term to the thermodynamic term $\beta T_o'/\overline{U}$ is expected to be $< (0.25\times 2)/10 = 5$\%. The time mean of this ratio is shown in figure \ref{fig:ratio_Q_eddy_Q_noeddy} for the winter of 2007, and as expected it generally does not exceed 5\%. In order for the dynamic term to be 10\% of the thermodynamic term in the $\beta$-model, $T_o'$ would need to be at least 4\,K. The small values of $T_o'$ appear to explain the small influence of the eddies from ERA5 compared to those found in other studies of idealized models or satellite observations \citep{foussard2019storm,ma2017revised}. In \citet{ma2017revised} the SST anomalies are approximately 3\,K while in \citet{foussard2019storm} many of the anomalies are nearly 6\,K (as shown in figure 3 of their paper) resulting in turbulent heat fluxes attributable to the mesoscale eddies on the order of 10 Wm$^{-2}$. The smaller anomalies considered here do not seem to provide enough leverage for the mesoscale eddies to exert a large influence on the air-sea heat flux.


\begin{figure}[tb]
\begin{tikzpicture}
    \draw (0, 0) node[inner sep=0]
    {\includegraphics[width=0.5\textwidth]{imgs/ABC_dynamic_L_250_fft_box3_beta_2007_abCDFAC_.pdf}};
    \draw (-2.3,-1) node[fill=white] {\scriptsize a)};
\end{tikzpicture}
\begin{tikzpicture}
    \draw (0, 0) node[inner sep=0] {\includegraphics[width=0.5\textwidth]{imgs/ABC_comp_To_L_250_fft_box3_beta_2007.pdf}};
    \draw (-2.3,-1) node[fill=white] {\scriptsize b)};
\end{tikzpicture}
    \caption{The sum of the rectified effect terms $\langle\overline{Q^\beta_3}+\overline{Q^\beta_4}\rangle$ (a) and the time mean $T_o'$ (b) for the winter of 2007.
    \label{fig:ABC_comp_To_2007}}
\end{figure}


\begin{figure}[tb]
    \centering
    \includegraphics[width=0.5\textwidth]{imgs/histogram_To_prime_L_250_fft_3_beta.pdf}
    \caption{The blue histogram shows the normalized frequency of the time mean $\langle T_o'\rangle$ of each winter, while the orange histogram shows the normalized frequency of the 6-hourly $T_o'$ for all winters.}
    \label{fig:histogram_Toprime}
\end{figure}

\begin{figure}[tb]
    \centering
    \includegraphics[width=0.5\textwidth]{imgs/ABC_ratio_L_250_fft_box3_beta_2007_abCDFAC_.pdf}
    \caption{The ratio of the time mean of the dynamic term $\langle \beta T_o' \Delta h \rangle$ to the time mean of the thermodynamic term $\langle \overline{U} \Delta h\rangle $ of the $\beta$-model for the winter of 2007.}
    \label{fig:ratio_Q_eddy_Q_noeddy}
\end{figure}



\subsection{The $\alpha\beta$-model}
As mentioned in section \ref{Sec:intro}, it is well documented that the drag coefficient is also affected by $T'_o$. Indeed, the possibility that such effect could be involved in setting the intensity of the western boundary currents themselves had been suggested in earlier studies \citep{behringer1979thermal}. To include this effect, we now include the impact of mesoscale SST anomalies on the drag coefficient (i.e. $C_D = C_D^{(ref)}(1+\alpha T_o')$) where $\alpha$ is a coupling coefficient capturing the effect of $T_o'$ on the exchange coefficient $C_D$. Continuing with equation \ref{eq:beta_model}, we obtain: 
\begin{align}
    Q^{\alpha\beta} = \rho_a C_D^{(ref)}(1+\alpha T_o')(\Ub+\beta T_o')\Delta h.
\end{align}
In this model the three optimised parameters are: $\beta$ as in the first model, $C_D^{(ref)}$, which is a reference or ``background'' value for the drag coefficient, and the eddy coupling coefficient $\alpha$, which when positive indicates an enhancement of the surface drag over a warm mesoscale eddy. 

This model also shows good agreement with the ERA5 data (see figure S2). The mean and standard deviations of the three model parameters $\alpha$, $\beta$, and $C_D$ across all years are $\alpha = $-3.3$\times10^{-4}$ $\pm$0.01\,K$^{-1}$, $\beta =$ 0.29$\pm$0.09,m\,s$^{-1}$K$^{-1}$, and $C_D^{(ref)} =$ (1.4$\pm$0.01)$\times10^{-3}$. In this model it appears that the $\beta$ term is sufficient to characterize the eddy-enhanced flux, as the optimised $\alpha$ coupling coefficient varies between positive and negative values in different years such that it effectively vanishes in the time mean across 2003-2018. As in the previous model, the long-time, large-spatial scale term dominates the other terms. The ratio of the dynamic terms to the thermodynamic terms in this model is $\left(\overline{U}\alpha \Top+\beta \Top + \alpha\beta \left(\Top\right)^2\right)/\overline{U}$. Using the average values of the model parameters, $\overline{U}$=10\,m/s, and $\Top<$2\,K again, the expected upper bound for this ratio is approximately 5.8\,\%. This bound is very similar to that from the $\beta$-model due to the small time-averaged value of $\alpha$, and the plot of this ratio for the winter of 2007 is shown in figure S4. \par 





The two models presented above are successful at representing the air-sea turbulent heat flux from the reanalysis fields and each isolate eddy contribution through different mechanisms using only one or two eddy coupling coefficients. The optimised values for all parameters and all the models, shown in figure S1, are found to be relatively stable from winter-to-winter. Neither the values of the eddy coupling coefficients (figure S1), nor the RMSEs (figures \ref{fig:RMSE_beta} and S3) appear to be sensitive to the degree of stability of the Kuroshio's path over a winter. The RMSEs calculated between the model and the ERA5 fields using the optimal value of the coupling coefficients is less than 20\,Wm$^{-2}$ for all models and all years considered. The goal of the $\alpha\beta$-model was to understand the relative importance of the different eddy coupling coefficients $\alpha$ and $\beta$, and the results suggest that the $\beta$ coefficient alone is able to capture the signal from the eddies. This roughly agrees with the findings reviewed by \citet{SMALL2008274} in which the sea surface drag enhancement of the surface wind stress was a smaller effect than the changes in atmospheric stability from the warm and cold eddies which drive near-surface winds. \par 







\section{Summary\label{Sec:Conclusion}}

The goal of this study was to quantify the oceanic eddy contribution to the winter time mean surface turbulent heat flux using reanalysis data. The correlation between the length of the SSHA contours, which was used as a proxy for eddy activity, and the winter time mean surface heat flux from the reanalysis data was also considered. No statistically significant correlations were detected when considering all years from 2003 through 2018, or when separately considering the years with stable Kuroshio profiles and the years with unstable Kuroshio profiles.\par 

To better understand the lack of a statistically significant correlation between SSHA and the turbulent heat flux, two models were constructed to isolate the influence of the ocean mesoscale eddies on the air-sea heat flux. The $\beta$-model considered the rectified effect to be expressed through an increase in the surface wind speed. The $\alpha\beta$-model incorporated a sea surface drag coupling mechanism in to the $\beta$-model, and the result was that on average this mechanism accounted for a much smaller effect than the surface wind coupling mechanism. Both models were able to well-represent the surface heat flux from the reanalysis data. Additionally, there was not much variance in the values of the eddy coupling coefficients or sea surface exchange coefficients between years, nor were the values of the coefficients sensitive to the stability of the Kuroshio. The results of separating out the dynamic from the thermodynamic terms showed that the mesoscale eddy contribution through the rectified effect in the reanalysis fields is small compared to the long-time, large-spatial scale mean fields. This disagrees with the hypothesis put forward in \citet{ma2016western} and \citet{ma2017revised}, possibly because the SST anomalies in ERA5 near the Kuroshio are too weak.\par 
Indeed, there are a couple recent studies that have suggested ERA5 may struggle to accurately estimate SST near western boundary currents. \citet{yang2021sea} conducted a detailed comparison of eight SST products, including ERA5 and OSTIA \citep{donlon2012operational} which is used within ERA5, for the years 2003-2018. Their results of the time-averaged difference of monthly SSTs between each product and the median of all the products showed that ERA5 generally agreed well with the ensemble median except for a few locations including regions near western boundary currents. Near the KE in particular, they found that on average ERA5 underestimated the SST by a small fraction of a degree Celsius compared to the ensemble median. While this study considered the total SST rather than the mean and anomaly separately, and compared ERA5 to other SST products rather than observations, the notable discrepancies in western boundary current regions which exhibit higher mesoscale eddy activity might indicate that ERA5 underestimates the magnitude of the SST fluctuations in this region. \citet{luo2020evaluation} compared the SST from ERA5 to ship-based radiometric measurements and found that while the two datasets generally agreed well, large air-sea temperature differences such as those that are present near the Gulf Stream, tend to increase the error between ERA5 and observations. While this study was focused on the Atlantic, the results show that near the Gulf Stream the discrepancy between the observations and ERA5 was as large as 1.5\,K.

Although the rectified effect from SST anomalies appears to be weak in the reanalysis data compared to other terms in the flux expansion, the eddy-enhancement is strong enough that the optimsed coupling coefficient $\beta$ in our study is similar to that estimated from satellite observations by \citet{ONeill2010}. A possible implication of this result is that ERA5 includes the effect of the mesoscale SST field on the surface windspeed accurately, but that this effect is just too small to make a significant contribution to the turbulent cooling of the Kuroshio. Alternatively, it could be that a straighter and most stable Kuroshio leads to more persistent and stronger temperature contrasts across it which sustain a larger cooling of the ocean. More work is needed to distinguish between these two different interpretations.






















\beginsupplement

% \appendix
% \section{Appendix}
% \setcounter{figure}{0}
% \renewcommand\thefigure{\Alph{section}.\arabic{figure}}
%\input{appendix}


\section*{ACKNOWLEDGEMENTS}
This study has been conducted using E.U. Copernicus Marine Service Information. \citep{ERA5_data} was downloaded from the Copernicus Climate Change Service (C3S) Climate Data Store. This work was supported by an MIT Office of Graduate Education Fellowship.

\section*{CONFLICT OF INTEREST}
The authors declare no conflict of interests.

\section*{SUPPORTING INFORMATION}

\noindent Figure S1: The optimised model parameters for both models and all years 2003-2018.\\
Figure S2: As in figure 2, except for the $\alpha\beta$ model.\\
Figure S3: As in figure 3, except for the $\alpha\beta$ model.\\
Figure S4: The ratio of the dynamic terms to the thermodynamic terms for the $\alpha\beta$ model.
%\printendnotes

% Submissions are not required to reflect the precise reference formatting of the journal (use of italics, bold etc.), however it is important that all key elements of each reference are included.
\bibliography{eddyFlux}

% \begin{biography}[example-image-1x1]{A.~One}
% Please check with the journal's author guidelines whether author biographies are required. They are usually only included for review-type articles, and typically require photos and brief biographies (up to 75 words) for each author.
% \bigskip
% \bigskip
% \end{biography}

% \graphicalabstract{example-image-1x1}{Please check the journal's author guildines for whether a graphical abstract, key points, new findings, or other items are required for display in the Table of Contents.}

\end{document}
