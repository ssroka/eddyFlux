
The goal of this study was to quantify the oceanic eddy contribution to the winter time mean surface turbulent heat flux using reanalysis data. The correlation between the length of the SSHA contours, which was used as a proxy for eddy activity, and the winter time mean surface heat flux from the reanalysis data was also considered. No statistically significant correlations were detected when considering all years from 2003 through 2018, or when separately considering the years with stable Kuroshio profiles and the years with unstable Kuroshio profiles.\par 

To better understand the lack of a statistically significant correlation between SSHA and the turbulent heat flux, two models were constructed to isolate the influence of the ocean mesoscale eddies on the air-sea heat flux. The $\beta$-model considered the rectified effect to be expressed through an increase in the surface wind speed. The $\alpha\beta$-model incorporated a sea surface drag coupling mechanism in to the $\beta$-model, and the result was that on average this mechanism accounted for a much smaller effect than the surface wind coupling mechanism. Both models were able to well-represent the surface heat flux from the reanalysis data. Additionally, there was not much variance in the values of the eddy coupling coefficients or sea surface exchange coefficients between years, nor were the values of the coefficients sensitive to the stability of the Kuroshio. The results of separating out the dynamic from the thermodynamic terms showed that the mesoscale eddy contribution through the rectified effect in the reanalysis fields is small compared to the long-time, large-spatial scale mean fields. This disagrees with the hypothesis put forward in \citet{ma2016western} and \citet{ma2017revised}, possibly because the SST anomalies in ERA5 near the Kuroshio are too weak.\par 
Indeed, there are a couple recent studies that have suggested ERA5 may struggle to accurately estimate SST near western boundary currents. \citet{yang2021sea} conducted a detailed comparison of eight SST products, including ERA5 and OSTIA \citep{donlon2012operational} which is used within ERA5, for the years 2003-2018. Their results of the time-averaged difference of monthly SSTs between each product and the median of all the products showed that ERA5 generally agreed well with the ensemble median except for a few locations including regions near western boundary currents. Near the KE in particular, they found that on average ERA5 underestimated the SST by a small fraction of a degree Celsius compared to the ensemble median. While this study considered the total SST rather than the mean and anomaly separately, and compared ERA5 to other SST products rather than observations, the notable discrepancies in western boundary current regions which exhibit higher mesoscale eddy activity might indicate that ERA5 underestimates the magnitude of the SST fluctuations in this region. \citet{luo2020evaluation} compared the SST from ERA5 to ship-based radiometric measurements and found that while the two datasets generally agreed well, large air-sea temperature differences such as those that are present near the Gulf Stream, tend to increase the error between ERA5 and observations. While this study was focused on the Atlantic, the results show that near the Gulf Stream the discrepancy between the observations and ERA5 was as large as 1.5\,K.

Although the rectified effect from SST anomalies appears to be weak in the reanalysis data compared to other terms in the flux expansion, the eddy-enhancement is strong enough that the optimsed coupling coefficient $\beta$ in our study is similar to that estimated from satellite observations by \citet{ONeill2010}. A possible implication of this result is that ERA5 includes the effect of the mesoscale SST field on the surface windspeed accurately, but that this effect is just too small to make a significant contribution to the turbulent cooling of the Kuroshio. Alternatively, it could be that a straighter and most stable Kuroshio leads to more persistent and stronger temperature contrasts across it which sustain a larger cooling of the ocean. More work is needed to distinguish between these two different interpretations.




















