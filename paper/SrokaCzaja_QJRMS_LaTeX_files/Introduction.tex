% it looks like QJRMS might be free if the page count is under 17.

The western boundary currents of the World Ocean provide an important heat source for the atmospheric storm-track \citep{hoskins1990existence}. Regions with strong currents like the Kuroshio Extension (KE) or Gulf Stream can be expected to modify the wind stress by as much as 20\%, which has significant implications for the air-sea heat flux in those regions \citep{Chelton2004}. This study focuses on the North Pacific and specifically the Kuroshio in winter, when the atmospheric eddy driven jet is closest to the path of the Kuroshio and its extension into the interior of the Pacific \citep{nakamura2004observed}. 




While large-scale (greater than 1000\,km) correlations between sea surface temperature (SST) and surface winds are generally negative, observations and simulations have confirmed a strong positive correlation at the mesoscale \citep{chelton2010coupled}. \citet{Chelton2004}, and subsequent studies including \citet{ONeill2010}, identified persistent, small-scale structures from high-resolution (25\,km) satellite measurements of surface winds and found that there were many features with characteristic wavelengths less than 30$^{\circ}$ in longitude and 10$^{\circ}$ in latitude which could be attributed to SST variations. The authors showed how the curl and divergence of the 4-year averaged wind field were directly proportional to the downwind and crosswind gradient of the SST field, respectively.\par

\citet{putrasahan2013isolating} isolated the influence of ocean mesoscale eddies on surface winds in the KE region by conducting companion simulations with a regional model where eddies are either resolved or filtered out. Their results showed that the eddies play an important role in driving the surface wind through two mechanisms, vertical mixing from alternating regions of boundary layer stability and sea level pressure anomalies. The strength of the former mechanism, which is measured by the magnitude of the linear scaling between the wind stress divergence (curl) and downwind (crosswind) SST gradient, was found to exhibit a strong seasonality being much larger in the winter, when the atmosphere is generally less stable, than the summer. The simulations exhibited a steeper linear relationship than was calculated from satellite observations, but the findings strongly supported the conclusion that the mesoscale eddies are an important driver of surface wind stress. With regard to the latter pressure-driven mechanism, which is measured by the linear scaling of the wind convergence and the sea level pressure Laplacian, the authors found that the presence of eddies significantly strengthened the ocean-atmosphere coupling compared to the eddy-free simulation.\par


It has recently been suggested that the meanders or eddies that develop on western boundary currents with characteristic length scales of a few 100\,km have a so called \textit{rectified effect} on the turbulent air-sea heat flux ($Q$) because a little bit more heat is lost over a warm anticyclone than is gained over a cold cyclone \citep{SMALL2008274,foussard2019storm}. This then was suggested to have a significant effect on the storm track, at least in the North Pacific \citep{ma2015improved,ma2017revised} and South Atlantic \citep{villas2015signature}. \citet{SMALL2008274} reviewed various physical mechanisms that facilitate air-sea heat flux from ocean eddies and fronts including: the destabilizing effect of air traveling over SST gradients, large eddies increasing the boundary layer depth, secondary circulations associated with spatial changes in the pressure-gradient force, and the way changes in ocean surface velocities can locally increase the surface stress. The source of SST anomalies near strong SST frontal regions like the KOE region is thought to be oceanic weather rather than atmospheric forcing \citep{bishop2017scale}.\par 

 

Isolating the effect of ocean mesoscale eddies on the turbulent heat fluxes from observations is difficult because, unlike in an atmospheric general circulation model (AGCM), one cannot separately analyse a time history of the heat flux with and without eddies, as was done in \citet{putrasahan2013isolating} or \citet{foussard2019storm} for example. In this paper we propose a new methodology which attempts to do so by developing simplified models of air-sea interactions tuned to realistic air-sea heat fluxes. The models use just one or two eddy coupling coefficients, so it is straightforward to evaluate the effect of the eddies on the time mean and spatially averaged $Q$. The first model represents the eddy-enhancement as an increase in the surface wind speed, and is based on the work from \citet{ONeill2010} who used wind vector and SST observations to estimate the wind vector magnitude and direction response to the persistent mesoscale eddies near western boundary currents. The authors found that air flow along a positive (negative) SST gradient led to an acceleration (deceleration) and anticyclonic (cyclonic) changes in the wind velocity; the magnitudes of changes in the wind velocity and direction were about 1-2\,ms$^{-1}$ and 4$^{\circ}$-8$^{\circ}$. The second model adds a mechanism to the first model such that the eddy-enhancement can also be represented through an increase in the sea surface drag following \citet{behringer1979thermal}. \citet{SMALL2008274} discuss several studies that consider the mechanism by which warmer SSTs lead to atmospheric instabilities which increase the local sea surface drag coefficient and therefore the stress. The emerging consensus appears to be that an increase in the sea surface drag with SST is a much smaller contribution to the increase in surface stress compared to other mechanisms like a pressure-gradient driven increase in the surface wind speed. 


  





The paper is structured as follows. We introduce the data used for the air-sea interaction analysis in section \ref{Sec:DataMethods}, and discuss the impact of the mesoscale ocean eddy activity on the time and spatial mean $Q$ in section \ref{Sec:Corr}. Section \ref{Sec:Param} introduces the interaction models and shows that the models are able to well-represent the time mean air-sea turbulent heat flux over the KE in winter. Finally, a summary is offered in section \ref{Sec:Conclusion}.  

