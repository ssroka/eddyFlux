\documentclass[12pt,a4paper]{article}
\usepackage[utf8]{inputenc}
\usepackage{amsmath}
\usepackage{amsfonts}
\usepackage{amssymb}
\usepackage{multicol}

% figures
\usepackage{graphicx}
\usepackage{subcaption}
\usepackage{wrapfig}
\usepackage[section]{placeins} % keeps floats inside this sections

% document propoerties format
\usepackage[margin=0.75in]{geometry}
\usepackage{fancyhdr}
\usepackage{indentfirst}

% new commands
\newcommand{\Vmag}[1]{\| \mathbf{#1}\|}
% newcommands for this document
\newcommand{\dudx}{\frac{\partial u}{\partial x}}
\newcommand{\dudy}{\frac{\partial u}{\partial y}}
\newcommand{\dvdx}{\frac{\partial v}{\partial x}}
\newcommand{\dvdy}{\frac{\partial v}{\partial y}}

\newcommand{\dTdx}{\frac{\partial T}{\partial x}}
\newcommand{\dTdy}{\frac{\partial T}{\partial y}}

\newcommand{\dTpdx}{\frac{\partial T'}{\partial x}}
\newcommand{\dTpdy}{\frac{\partial T'}{\partial y}}

% bibliography
\usepackage[round]{natbib}

\begin{document}


\section{Idealized Front}
Create the setup from \citet{Foussard2019} and consider

\begin{figure}[h!]
\begin{subfigure}[t]{0.5\textwidth}
\includegraphics[width=\textwidth]{imgs_channel/SST.png}
\caption{Manufactured SST temperature }
\end{subfigure}
\begin{subfigure}[t]{0.5\textwidth}
\includegraphics[width=\textwidth]{imgs_channel/SST_prime.png}
\caption{Manufactured SST temperature anomaly from zonal mean}
\end{subfigure}
\end{figure}

\begin{align}
C_D &= C_D^*(1+\alpha T_o')\\
Q_s &= \rho_a c_p C_D \Vmag{U} (T_o-T_a)\\
Q_L &= \rho_a L_v C_D \Vmag{U} (q_o^*-q_a)
\end{align}
where the free parameters are, $T_o-T_a$ $(\Delta T)$, $\alpha$, and relative humidity ($RH$).

A wind field is applied (see three cases below) and the wind stress is calculated as $\tau_{xy} = \rho_a C_D \| \vec{u} \|\vec{u}$. Let $\hat{u}$ be the unit vector in the direction of the wind at each point. The along-wind ($aw$) and cross-wind ($cw$) components of the SST gradients are
\begin{align}
\nabla_{aw} T_o &= (\nabla T_o \cdot \hat{u}) \cdot \hat{u}\\
\nabla_{cw} T_o &= \nabla T_o - \left( \left(\nabla T_o \cdot \hat{u}\right) \cdot \hat{u} \right)
\end{align}

\subsection{Constant Pressure and Velocity ($\vec{u} = (\bar{u},0)$, $p = p_0$)}

With a constant pressure and velocity field the curl of the stress is directly proportional to the cross-wind SST gradient, and  the divergence of the stress is directly proportional to the down-wind SST gradient (see Section \ref{sec:A_constant}) at each point. For a particular setup with $C_D^* = 1E-3$, $\rho_a = 1.2$ kg/m$^{-3}$, $\alpha = 1E-3$, $\bar{u} = 4$ m/s, the slope of the curl of the stress (y-axis) to the cross-wind SST gradient (x-axis) is 1.92E-5 as seen in Figure \ref{Fig:constant}.

\begin{figure}[h!]
\centering
\begin{subfigure}[t]{0.6\textwidth}
\includegraphics[width=\textwidth]{imgs_channel/P_u_constant.png}
\caption{Manufactured SST temperature}
\end{subfigure}
\begin{subfigure}[t]{\textwidth}
\includegraphics[width=\textwidth]{imgs_channel/slope_stress_vs_SSTgrad_constant.png}
\caption{Sample of 10 points in blue circles (however all points in the domain fall on red line which has a slope of 1.92E-5)}
\end{subfigure}
\caption{\label{Fig:constant}}
\end{figure}

\subsection{Chelton Coupling Coefficient ($u = \overline{u} + \gamma T'$, $p = \overline{p}$)}
From \citep{Chelton2004}, the coupling coefficient is between 0.2 and 0.44. A detailed derivation is in Section \ref{sec:A_uT}, but the conclusion that if $\gamma T'$ is small with respect to $\bar{u}$, then the slope of the divergence plotted against the "down-wind" SST gradient will be $C_D^*\rho_a 2\gamma\bar{u}$, which for this setup would be $C_D^*\rho_a 2\gamma\bar{u} = 0.0019$ and the slope of the best fit line for this setup is 0.002 as shown in Figure \ref{Fig:uT}. \\

\begin{figure}[h!]
\centering
\begin{subfigure}[t]{0.6\textwidth}
\includegraphics[width=\textwidth]{imgs_channel/P_u_uT.png}
\end{subfigure}
\begin{subfigure}[t]{\textwidth}
\includegraphics[width=\textwidth]{imgs_channel/slope_stress_vs_SSTgrad_uT.png}
\caption{All points in the domain are represented by blue circles, the slope of the divergence-down-wind gradient fitted line is 0.002.}
\end{subfigure}
\caption{\label{Fig:uT}}
\end{figure}

\subsection{Baroclinic Wave ($u = u_{\text{geostrophic}}$ )}


\begin{figure}[h!]
\centering
\begin{subfigure}[t]{0.6\textwidth}
\includegraphics[width=\textwidth]{imgs_channel/P_u_baroclinicWave.png}
\caption{The pressure field is two Gaussians creating a wavelength of $\lambda = 4000$km.}
\end{subfigure}
\begin{subfigure}[t]{\textwidth}
\includegraphics[width=\textwidth]{imgs_channel/slope_stress_vs_SSTgrad_bw.png}
\caption{All points in the domain are represented by blue circles}
\end{subfigure}
\caption{\label{Fig:uT}}
\end{figure}


\newpage

\section{ERA5}


\subsection{Which years to pick?}

\begin{figure}[h!]
\includegraphics[width=\textwidth]{imgs_channel/Qiu_Chen_2010_Kuroshio_and_KuroshioExt_paths.png}
\caption{\citet{Aiu2010}}
\end{figure}

\begin{table}[h!]
\begin{tabular}{llll}
$\bar{\bullet}$ & zonal average & $\bullet'$ & zonal anomaly\\
$\langle\bullet\rangle$ & temporal average & $C_D^*$ & 1E-3\\
$T_o$ & Sea Surface Temperature & $T_a$ & 2m Temperature\\
$q_o$ & saturated specific humidity at the local SST & $a_a$ & specific humidity at the 2m temperature \\
$RH$ & relative humidity at 2m & $T_d$ & dew point temperature at 2m\\
$c_p$ & specific heat of air & $L_v$ & latent heat of vaporization 
\end{tabular}

\end{table}

ERA5 has 0.25$^{\circ}$ resolution.

\begin{align}
C_D^{s,L} = C_D^*(1+\alpha_{s,L} T')\\
Q_s = \rho_a c_p C_D^s \Vmag{U} (T_o-T_a)\\
Q_L = \rho_a L_v C_D^L \Vmag{U} (q_o^*-q_a)
\end{align}

\subsection{Full Atmosphere Full Ocean}\label{sec:full}
Use $Q_s$ and $Q_L$ from ERA5 (along with the SST, 2m dewpoint temperature, 2m temperature, surface pressure, and 10m horizontal wind speed) to calculate $\alpha$ at every point in space and time. 

\begin{align}
T_a &=\overline{T_a} + T_a' \\
T_d &=\overline{T_d} + T_d' \\
p_0 &=\overline{p_0} + p_0' \\
u_{10} &= \overline{u_{10}} + u_{10}' \\
v_{10} &= \overline{v_{10}} + v_{10}'\\
\Vmag{U} &= \sqrt{u_{10}^2 + v_{10}^2}\\
q_o &= q_o(\overline{T_o}+T_o',\overline{p_0}+p_0')\\
q_a &= q_a(\overline{T_a}+T_a',\overline{T_d}+T_d',\overline{p_0}+p_0')
\end{align}

\begin{align}
\alpha_s = \frac{1}{T'}\left( 1-\frac{Q_S}{\rho_a c_p C_D^* \Vmag{U} (T_o-T_a)} \right)\\
\alpha_L = \frac{1}{T'}\left( 1-\frac{Q_S}{\rho_a L_v C_D^* \Vmag{U} (q_o-q_a)} \right)
\end{align}

\begin{figure}[h!]
\centering
\begin{subfigure}[t]{0.49\textwidth}
\includegraphics[width=\textwidth]{imgs_channel/era5_QL_03.png}
\end{subfigure}
\begin{subfigure}[t]{0.49\textwidth}
\includegraphics[width=\textwidth]{imgs_channel/era5_slhf_03.png}
\end{subfigure}
\caption{The mean SLHF from Equation \ref{Eq:QL} and directly from the reanalysis data}
\end{figure}

\begin{figure}[h!]
\centering
\begin{subfigure}[t]{0.49\textwidth}
\includegraphics[width=\textwidth]{imgs_channel/era5_Qs_03.png}
\end{subfigure}
\begin{subfigure}[t]{0.49\textwidth}
\includegraphics[width=\textwidth]{imgs_channel/era5_sshf_03.png}
\end{subfigure}
\caption{The mean SSHF from Equation \ref{Eq:Qs} and directly from the reanalysis data}
\end{figure}

\begin{figure}[h!]
\centering
\begin{subfigure}[t]{0.49\textwidth}
\includegraphics[width=\textwidth]{imgs_channel/era5_QL_07.png}
\end{subfigure}
\begin{subfigure}[t]{0.49\textwidth}
\includegraphics[width=\textwidth]{imgs_channel/era5_slhf_07.png}
\end{subfigure}
\caption{The mean SLHF from Equation \ref{Eq:QL} and directly from the reanalysis data}
\end{figure}

\begin{figure}[h!]
\centering
\begin{subfigure}[t]{0.49\textwidth}
\includegraphics[width=\textwidth]{imgs_channel/era5_Qs_07.png}
\end{subfigure}
\begin{subfigure}[t]{0.49\textwidth}
\includegraphics[width=\textwidth]{imgs_channel/era5_sshf_07.png}
\end{subfigure}
\caption{The mean SSHF from Equation \ref{Eq:Qs} and directly from the reanalysis data}
\end{figure}


\begin{figure}[h!]
\centering
\begin{subfigure}[t]{0.49\textwidth}
\includegraphics[width=\textwidth]{imgs_channel/era5_aL_03.png}
\end{subfigure}
\begin{subfigure}[t]{0.49\textwidth}
\includegraphics[width=\textwidth]{imgs_channel/era5_as_03.png}
\end{subfigure}
\caption{The mean $\alpha$ values (i.e. $\alpha$ is calculated every 6 hours then point-wise averaged in time). The color contours are limited to $\pm$5 since at some points the value of $\alpha$ contains "Inf" values. }
\end{figure}

\begin{figure}[h!]
\centering
\includegraphics[width=\textwidth]{imgs_channel/alpha_smooth.png}
\caption{using a cut-off and box-car smoothing the $\alpha$ fields.}
\end{figure}

\begin{figure}[h!]
\centering
\includegraphics[width=\textwidth]{imgs_channel/alpha_smooth_07.png}
\caption{using a cut-off and box-car smoothing the $\alpha$ fields.}
\end{figure}


\begin{figure}[h!]
\centering
\begin{subfigure}[t]{0.49\textwidth}
\includegraphics[width=\textwidth]{imgs_channel/era5_CDL_03.png}
\end{subfigure}
\begin{subfigure}[t]{0.49\textwidth}
\includegraphics[width=\textwidth]{imgs_channel/era5_CDs_03.png}
\end{subfigure}
\caption{The drag coefficients using ERA5 SST' and smoothed $\alpha$ fields.}
\end{figure}

\subsection{Full Atmosphere, Smooth Ocean}
Compute the flux from a smooth sea surface where $T_o' = 0$.

\begin{align}
T_a &=\overline{T_a} + T_a' \\
T_o &=\overline{T_o}  \\
T_d &=\overline{T_d} + T_d' \\
p_0 &=\overline{p_0} + p_0' \\
u_{10} &= \overline{u_{10}} + u_{10}' \\
v_{10} &= \overline{v_{10}} + v_{10}'\\
\Vmag{U} &= \sqrt{u_{10}^2 + v_{10}^2}\\
q_o &= q_o(\overline{T_o},\overline{p_0}+p_0')\\
q_a &= q_a(\overline{T_a}+T_a',\overline{T_d}+T_d',\overline{p_0}+p_0')
\end{align}

\begin{align}
Q_s = \rho_a c_p  C_D^* \Vmag{U} (T_o-T_a) \label{Eq:Qs}\\
Q_L = \rho_a L_v  C_D^* \Vmag{U} (q_o^*-q_a) \label{Eq:QL}
\end{align}


\begin{figure}[h!]
\centering
\begin{subfigure}[t]{0.49\textwidth}
\includegraphics[width=\textwidth]{imgs_channel/era5_QL_so_03.png}
\end{subfigure}
\begin{subfigure}[t]{0.49\textwidth}
\includegraphics[width=\textwidth]{imgs_channel/era5_Qs_so_03.png}
\end{subfigure}
\end{figure}


\subsection{Vanished Anomaly}
Compute the flux from a smooth sea surface where $T' = 0$.

\begin{align}
T_a &=\overline{T_a} \\
T_o &=\overline{T_o}  \\
T_d &=\overline{T_d} \\
p_0 &=\overline{p_0} \\
u_{10} &= \overline{u_{10}} \\
v_{10} &= \overline{v_{10}} \\
\Vmag{U} &= \sqrt{u_{10}^2 + v_{10}^2}\\
q_o &= q_o(\overline{T_o},\overline{p_0})\\
q_a &= q_a(\overline{T_a},\overline{T_d},\overline{p_0})
\end{align}


\begin{align}
Q_s = \rho_a c_p  C_D^* \Vmag{U} (T_o-T_a)\\
Q_L = \rho_a L_v  C_D^* \Vmag{U} (q_o^*-q_a)
\end{align}


\begin{figure}[h!]
\centering
\begin{subfigure}[t]{0.49\textwidth}
\includegraphics[width=\textwidth]{imgs_channel/era5_QL_va_03.png}
\end{subfigure}
\begin{subfigure}[t]{0.49\textwidth}
\includegraphics[width=\textwidth]{imgs_channel/era5_Qs_va_03.png}
\end{subfigure}
\end{figure}


\begin{figure}[h!]
\centering
\begin{subfigure}[t]{0.49\textwidth}
\includegraphics[width=\textwidth]{imgs_channel/era5_QL_va_07.png}
\end{subfigure}
\begin{subfigure}[t]{0.49\textwidth}
\includegraphics[width=\textwidth]{imgs_channel/era5_Qs_va_07.png}
\end{subfigure}
\end{figure}


\section{Idealized Front vs ERA 5}

\begin{figure}[h!]
\centering
\begin{subfigure}[t]{0.49\textwidth}
\includegraphics[width=\textwidth]{imgs_channel/SST.png}
\end{subfigure}
\begin{subfigure}[t]{0.49\textwidth}
\includegraphics[width=\textwidth]{imgs_channel/era5_SST_03.png}
\end{subfigure}
\caption{The mean SST from reanalysis is qualitatively similar to that from the idealized from in terms of the range of temperatures, }
\end{figure}

\begin{figure}[h!]
\centering
\begin{subfigure}[t]{0.49\textwidth}
\includegraphics[width=\textwidth]{imgs_channel/SST_prime.png}
\end{subfigure}
\begin{subfigure}[t]{0.49\textwidth}
\includegraphics[width=\textwidth]{imgs_channel/era5_SST_prime_03.png}
\end{subfigure}
\caption{The mean anomaly (i.e. calculating the anomaly from the zonal mean every 6 hours then averaging all the anomalies together) from the ERA5 reanalysis data has smaller amplitudes than that from the idealized front.}
\end{figure}

\begin{figure}[h!]
\centering
\begin{subfigure}[t]{0.49\textwidth}
\includegraphics[width=\textwidth]{imgs_channel/era5_RH_03.png}
\caption{The RH from ERA5 is much larger in general than the 80\% assumed by the idealized front experiment.}
\end{subfigure}
\begin{subfigure}[t]{0.49\textwidth}
\includegraphics[width=\textwidth]{imgs_channel/era5_DT_03.png}
\caption{The DT from ERA5 is much larger in general than the 0.5 assumed by the idealized front experiment.}
\end{subfigure}

\end{figure}

\section{Compare Smoothing SST with boxcar to zonal average}

\begin{figure}[h!]
\centering
\begin{subfigure}[t]{0.49\textwidth}
\includegraphics[width=\textwidth]{imgs_channel/era5_SST_prime_zonal_03.png}
\end{subfigure}
\begin{subfigure}[t]{0.49\textwidth}
\includegraphics[width=\textwidth]{imgs_channel/era5_SST_prime_box_03.png}
\end{subfigure}
\caption{The median of $SST'$ for 2003 where $\overline{SST}$ is computed with a zonal (left) vs box (right) average.}
\end{figure}

\begin{figure}[h!]
\centering
\begin{subfigure}[t]{0.49\textwidth}
\includegraphics[width=\textwidth]{imgs_channel/era5_SST_prime_zonal_07.png}
\end{subfigure}
\begin{subfigure}[t]{0.49\textwidth}
\includegraphics[width=\textwidth]{imgs_channel/era5_SST_prime_box_07.png}
\end{subfigure}
\caption{The median of $SST'$ for 2003 where $\overline{SST}$ is computed with a zonal (left) vs box (right) average.}
\end{figure}



\begin{figure}[h!]
\centering
\begin{subfigure}[t]{\textwidth}
\includegraphics[width=\textwidth]{imgs_channel/alpha_median_box_03.png}
\end{subfigure}
\begin{subfigure}[t]{\textwidth}
\includegraphics[width=\textwidth]{imgs_channel/alpha_median_zonal_03.png}
\end{subfigure}
\caption{The median of $\alpha$'s in time per spatial location using the full ERA5 fluxes for the 2003 winter which had low variability in the Kuroshio. The boxcar clearly better preserves the smaller scale features.}
\end{figure}

\begin{figure}[h!]
\centering
\begin{subfigure}[t]{\textwidth}
\includegraphics[width=\textwidth]{imgs_channel/alpha_median_box_07.png}
\end{subfigure}
\begin{subfigure}[t]{\textwidth}
\includegraphics[width=\textwidth]{imgs_channel/alpha_median_zonal_07.png}
\end{subfigure}
\caption{The median of $\alpha$'s in time per spatial location using the full ERA5 fluxes for the 2003 winter which had low variability in the Kuroshio. The boxcar clearly better preserves the smaller scale features.}
\end{figure}

\section{Using a constant $\alpha$ might introduce too much error}

\begin{figure}[h!]
\centering
\begin{subfigure}[t]{\textwidth}
\includegraphics[width=\textwidth]{imgs_channel/alpha_const_error_box_03.png}
\end{subfigure}
\begin{subfigure}[t]{\textwidth}
\includegraphics[width=\textwidth]{imgs_channel/alpha_const_error_box_07.png}
\end{subfigure}
\caption{Each quantity -the flux from ERA5 (left), the flux calculation with constant $\alpha$ and spatially and temporally varying ERA5 fields of SST and air properties (middle), and the difference between these two flux calculations (right) - is calculated at each 6hour point then the median of these is displayed here. The constant $\alpha$ is chosen by taking the median of all calculated $\alpha$'s.}
\end{figure}

\section{Using a linear $\alpha$ also likely has way too much error}

\begin{figure}[h!]
\centering
\includegraphics[width=\textwidth]{imgs_channel/alpha_linear_noW.png}
\caption{There appears to be some correlation}
\end{figure}

\begin{figure}[h!]
\centering
\includegraphics[width=\textwidth]{imgs_channel/W_Gaus.png}
\caption{There appears to be some correlation}
\end{figure}

\begin{figure}[h!]
\centering
\includegraphics[width=\textwidth]{imgs_channel/alpha_linear_Gaus.png}
\caption{There appears to be some correlation}
\end{figure}

\begin{figure}[h!]
\centering
\includegraphics[width=\textwidth]{imgs_channel/alpha_linear_SST.png}
\caption{There appears to be some correlation}
\end{figure}


\section{Is the $\mathbb{V}(SST)$ correlated with the average flux?}

\begin{figure}[h!]
\centering
\includegraphics[width=\textwidth]{imgs_channel/var_vs_flux.png}
\caption{There appears to be some correlation}
\end{figure}

\begin{figure}[h!]
\centering
\includegraphics[width=\textwidth]{imgs_channel/var_vs_flux_SST_prime.png}
\caption{There appears to be some correlation}
\end{figure}

\begin{figure}[h!]
\centering
\includegraphics[width=\textwidth]{imgs_channel/var_vs_flux_SST_prime_med.png}
\caption{There appears to be some correlation}
\end{figure}

\appendix
\section{Proving the curl/divergence and cross-wind/down-wind gradient relationships}
\subsection*{Variables:}
\begin{table}[h!]
\begin{tabular}{ll}
 $C_D^*$ &  Reference drag coefficient (1E-3) \\
 $\rho_a$ &  air density [kg m$^{-3}$] \\
 $T_C$ & temperature field in ($y$) without eddies in [K]\footnote{The CTRL field in Foussard et al. 2019}  \\
 $T'$ & temperature perturbation (eddies only) in ($x,y$) in [K]\footnote{The EDDY field in Foussard et al. 2019}  \\
  $T$ & total temperature field ($=T_C + T'$) in ($x,y$) in [K]  \\
 $\vec{u}$ & velocity vector with components $(u,v)$ in [m/s] \\
 $x,y$ & spatial coordinates [m]
\end{tabular}
\end{table}

\subsection{stress calculations}
For tuning parameter $\alpha$, the surface stress is 
\begin{align*}
\tau = \rho_a C_D^*(1+\alpha T')\vec{u}\lVert u \rVert
\end{align*}

The divergence of the stress is 

\begin{align*}
\begin{split}
\nabla \cdot \tau = C_D^* \rho_a \left( \frac{(1+\alpha T')(4\dudx u^3 + 2 \dvdx u^2 v + 2 \dudx u v^2)}{2\sqrt{u^4+u^2v^2}}+ \frac{(1+\alpha T')(4\dvdy v^3 + 2 \dudy u v^2 + 2 \dvdy u^2)}{2\sqrt{v^4+u^2v^2}}\right.+...\\
\left. \alpha\dTpdx\sqrt{u^4+u^2v^2}-\alpha\dTpdy\sqrt{v^4+u^2v^2} \right)
\end{split}
\end{align*}

The curl of the stress is 

\begin{align*}
\begin{split}
\nabla \times \tau = C_D^* \rho_a \left( \frac{(1+\alpha T')(4\dudy u^3 + 2 \dvdy u^2 v + 2 \dudy u v^2)}{2\sqrt{u^4+u^2v^2}}- \frac{(1+\alpha T')(4\dvdx v^3 + 2 \dudx u v^2 + 2 \dvdx u^2)}{2\sqrt{v^4+u^2v^2}}\right.+...\\
\left. \alpha\dTpdy\sqrt{u^4+u^2v^2}-\alpha\dTpdx\sqrt{v^4+u^2v^2} \right)
\end{split}
\end{align*}

The gradient of the sea surface temperature in the "down-wind" or the "downwind" direction is

\begin{align*}
\nabla SST_{\parallel} = \left( \frac{u\left( \frac{\dTdx u}{\sqrt{u^2+v^2}} + \frac{\dTdy v}{\sqrt{u^2+v^2}}\right)}{\sqrt{u^2+v^2}} ,  \frac{v\left( \frac{\dTdx u}{\sqrt{u^2+v^2}} + \frac{\dTdy v}{\sqrt{u^2+v^2}}\right)}{\sqrt{u^2+v^2}} \right)
\end{align*}

The gradient of the sea surface temperature in the "cross-wind" direction is

\begin{align*}
\nabla SST_{\bot} = \left(\dTdx- \frac{u\left( \frac{\dTdx u}{\sqrt{u^2+v^2}} + \frac{\dTdy v}{\sqrt{u^2+v^2}}\right)}{\sqrt{u^2+v^2}} , \dTdy- \frac{v\left( \frac{\dTdx u}{\sqrt{u^2+v^2}} + \frac{\dTdy v}{\sqrt{u^2+v^2}}\right)}{\sqrt{u^2+v^2}} \right)
\end{align*}


\subsection{In the limit that $\vec{u} = (\bar{u},0)$ everywhere}\label{sec:A_constant}

This means that $\dudx = \dudy = \dvdx = \dvdy = v = 0$ and $u=\bar{u}$

So the above quantities become:\\
Divergence:
\begin{align*}
\nabla \cdot \tau = C_D^* \rho_a \alpha\bar{u}^2\dTpdx
\end{align*}
Down-wind gradient:
\begin{align*}
\nabla SST_{\parallel} &= \left( \dTdx , 0 \right) \\
&= \left( \frac{\partial}{\partial x}(T_C + T') , 0) \right)\\
&= \left( \dTpdx, 0 \right)
\end{align*}

Which, when plotted against eachother, will have a slope of $ C_D^* \rho_a \alpha\bar{u}^2$.\\
\vspace*{0.5in}

Curl:
\begin{align*}
\nabla \times \tau = -C_D^* \rho_a \alpha\bar{u}^2\dTpdy
\end{align*}
Cross-wind gradient:
\begin{align*}
\nabla SST_{\bot} &= \left(\dTdx- \dTpdx , \dTdy- 0 \right)\\
\nabla SST_{\bot} &= \left(0 , \frac{\partial}{\partial y}(T_C + T') \right)\\
\nabla SST_{\bot} &= \left(0 , \frac{\partial}{\partial y}T_C+ \frac{\partial}{\partial y}T' \right)
\end{align*}

Which, when plotted against eachother, will have a slope of approximately $ -C_D^* \rho_a \alpha\bar{u}^2$ if $\frac{\partial T_C}{\partial y}$ is small relative to $\dTpdy$.

\subsection{In the limit that the velocity field is determined by a coupling coefficient }\label{sec:A_uT}
 $\vec{u} = (\bar{u} + T' \gamma, 0)$ for a coupling coefficient $\gamma$ and the drag coefficient is nolonger a function of temperature (i.e. $\alpha = 0$)

The divergence of the stress is now

\begin{align*}
\nabla \cdot \tau =& C_D^* \rho_a \left( \frac{(4\dudx u^3)}{2u^2} \right)\\
\nabla \cdot \tau =& C_D^* \rho_a \left( 2u\dudx \right)\\
\nabla \cdot \tau =& C_D^* \rho_a \left( 2u\gamma\dTpdx  \right)\\
\nabla \cdot \tau =& C_D^* \rho_a \left( 2\gamma\dTpdx (\bar{u} + \gamma T') \right)
\end{align*}

and the gradient of the sea surface temperature in the "downwind" direction is 

\begin{align*}
\nabla SST_{\parallel} = \left( \dTpdx, 0 \right)
\end{align*}
If $\gamma T'$ is small with respect to $\bar{u}$, then the slope of the divergence plotted against the "down-wind" SST gradient will be $C_D^*\rho_a 2\gamma\bar{u}$. \\

The curl of the stress is now

\begin{align*}
\nabla \times \tau =& C_D^* \rho_a \left( \frac{(4\dudy u^3 )}{2u^2} \right)\\
\nabla \times \tau =& C_D^* \rho_a \left(  2u\dudy \right)\\
\nabla \times \tau =& C_D^* \rho_a \left(  2u\gamma\dTpdy \right)\\
\nabla \times \tau =& C_D^* \rho_a \left( 2\gamma\dTpdy (\bar{u} + \gamma T') \right)
\end{align*}

and the gradient of the sea surface temperature in the "cross-wind" direction is 

\begin{align*}
\nabla SST_{\bot} &= \left(0 , \frac{\partial}{\partial y}T_C+ \frac{\partial}{\partial y}T' \right)
\end{align*}

so in the limit that  $\gamma T'$ is small with respect to $\bar{u}$ and $\frac{\partial}{\partial y}T_C$ is small with respect to $\frac{\partial}{\partial y}T'$, the resulting slope of these two quantities plotted against eachother would also be $C_D^* \rho_a 2\gamma\bar{u}$. \\


\bibliographystyle{apa}
\bibliography{/Users/ssroka/MIT/Research/EmanuelGroup/References/bibtex/all_of_everything}


\end{document}